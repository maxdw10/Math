\documentclass{amsart}
\usepackage{graphicx} % Required for inserting images
\usepackage[margin=1.4in, top=1.2in, bottom=1.2in]{geometry}
\usepackage{amsmath, amssymb, amsfonts}
\usepackage{tikz}
\usetikzlibrary{cd}
\usetikzlibrary{fit,shapes.geometric}
\tikzset{%  
    mdot/.style={draw, circle, fill=black},
    mset/.style={draw, ellipse, very thick},
}
\usepackage{enumitem}




\title{Algebra 245 Lecture 3}
\author{Notes by Max Weinstein}
\date{January 2024}

\begin{document}

\theoremstyle{plain}                    
\newtheorem{theorem}{Theorem}[section]
\newtheorem{lemma}[theorem]{Lemma}
\newtheorem{proposition}[theorem]{Proposition}
\newtheorem{corollary}[theorem]{Corollary} 

\theoremstyle{definition}
\newtheorem{definition}[theorem]{Definition}
\newtheorem{example}[theorem]{Example}
% \theoremstyle{remark}
\newtheorem{remark}[theorem]{Remark}


\newcommand{\nn}{\mathbb N}
\newcommand{\zz}{\mathbb Z}
\newcommand{\qq}{\mathbb Q}
\newcommand{\rr}{\mathbb R}
\newcommand{\cc}{\mathbb C}
\newcommand{\dd}{\mathbb D} 
\newcommand{\ff}{\mathbb F} 




\maketitle

\section{\textbf{Free Groups}}

If $A$ a set, then $F(A)$ is the free group generated by the elements of $S$ and we get a correspondence between sets and groups.\\
\begin{example}
    The empty set corresponds with the trivial group $$\varnothing \longleftrightarrow \{1\}$$ and a set with one element corresponds to a group generated by that element (with no relations) $$\{a\} \longleftrightarrow \langle a\rangle \cong \mathbb{Z}$$
\end{example}
\hfill
\begin{definition}
Define a category of maps $j$ from the set $A$ to any group $G$ with morphisms $\sigma$ group homomorphisms that make the diagram commute:
    \begin{center}
  \begin{tikzcd}
    A \arrow[r, "j_1"] \arrow[dr, "j_2"]& G \arrow[d, "\sigma"] \\
    & H
  \end{tikzcd}
\end{center}
$F(A)$ is defined to be the initial object in this category (Like final objects, initial objects are unique up to isomorphism).


\end{definition}
\hfill

\textbf{Why do Free Groups exist?}\\
$A$ a finite set, $A'$ the formal "inverses" of elements of $A$, then define $W(A)$ to be the set of words in $A\cup A'$
\begin{example}
    $A = \{a,b,c\}$, $A'=\{a^{-1}, b^{-1}, c^{-1}\}$, then $aaab^{-1}bc \in W(A)$
\end{example}

\begin{definition}
    $r:W(A)\to W(A)$ takes words in $W(A)$ and removes the first occurrence of $ll^{-1}$ for any letter $l$, so $r(aaa^{-1}bb^{-1})=abb^{-1}$, and fully reduced words remain unchanged, i.e. $r(abc)=abc$. The maximum number of iterations of $r$ to reduce a word of length $n$ is $\lfloor \frac{n}{2}\rfloor$, so now we define $R:W(A)\to W(A)$ to be the map taking words $w$ of length $n$ to $r^{\lfloor \frac{n}{2}\rfloor}(w)$, the fully reduced version of $w$.
    \\
    \\
    In this context, $F(A)$ is defined to be the group $(R(W(A)), \cdot)$ where $w\cdot w' = R(ww')$
\end{definition}
Proof that $F(A)$ is a group:\\
- Associativity: $w\cdot (w'\cdot w'')=R(wR(w'w''))=R(ww'w'')=R(R(ww')w'')=(w\cdot w')\cdot w''$\\
- Identity: the empty word\\
- Inverse: write the anti-word, i.e. $(abba^{-1})^{-1}=ab^{-1}b^{-1}a^{-1}$
\pagebreak

We can now finish the diagram:
\begin{center}
  \begin{tikzcd}
    A \arrow[r, "i"] \arrow[dr, "j"]& F(A) \arrow[d, "\sigma"] \\
    & G
  \end{tikzcd}
\end{center}
where $i(a)=a\in F(A)$ as a word, $j$ any map from the set $A$ to a group $G$, and $\sigma$ the unique homomorphism making the diagram commute. $\sigma(a) = j(a)$ and $\sigma(a^{-1})=j(a)^{-1}$.
\\
\section{\textbf{Subgroups}}
\hfill
\\
We write $(H,\cdot)<(G,\cdot)$ and say $(H,\cdot)$ is a subgroup of $(G,\cdot)$ if $H$ is a subset of $G$ which is itself a group under the same operation as $G$, i.e. $i: (H,\cdot)\hookrightarrow{} (G,\cdot)$ is a homomorphism.
\begin{lemma}
    Let $H$ be a non-empty subset of $G$, then $H$ is a subgroup iff $\forall a,b\in H$, $ab^{-1}\in H$\\
    \\
    Proof: $\Rightarrow$ If $H$ is a subgroup, then $H$ is a group so $a,b\in H$ implies $b^{-1}\in H$ and $ab^{-1}\in H$\\
    \\
    $\Leftarrow$ Suppose $ab^{-1}\in H$ whenever $a,b\in H$, then in particular for $h\in H, hh^{-1}=e\in H$. $e,h\in H$ implies $eh^{-1}=h^{-1}\in H$, and finally if $g,h\in H$, $g(h^{-1})^{-1}=gh\in H$, so $H$ is a group and therefore a subgroup of $G$
\end{lemma}
\hfill
\section{\textbf{Normal Subgroups}}
\hfill
\begin{definition}
    $N<G$ is normal if $\forall h\in N,$ and $\forall g\in G$, $ghg^{-1}\in N$. Equivalently, $N$ is normal if $gNg^{-1}=N \hspace{10pt} \forall g\in G$, where $gN = \{gh \hspace{2pt}| \hspace{2pt}h\in N\}.$ 
\end{definition}
\hfill
\begin{definition}
    $\phi : G\to H$ is a homomorphism if $\phi(g_1g_2)=\phi(g_1)\phi(g_2)$. The kernel of $\phi$ is the set $\text{Ker}(\phi)=\{g\in G \hspace{2pt} | \hspace{2pt} \phi(g)=e$\} 
\end{definition}
\hfill
\begin{lemma}
    $\text{Ker}(\phi)$ is a normal subgroup\\
    \\
    Proof: First notice that $\phi$ is a homomorphism, so $\phi(e)=e$ and $\text{Ker}(\phi)$ is always non-empty. Then let $g,h\in \text{Ker}(\phi)$, and observe $\phi(gh^{-1})=\phi(g)\phi(h^{-1})=e\phi(h)^{-1}=e^{-1}=e$, which means $gh^{-1}\in Ker(\phi)$ and $Ker(\phi)$ is a subgroup by Lemma 2.1.\\
    \\
    $\phi(h)=id$ implies $\phi(ghg^{-1})=\phi(g)\phi(h)\phi(g^{-1})=\phi(g)e\phi(g^{-1})=\phi(g)\phi(g^{-1})=e$, so $ghg^{-1}\in\text{Ker}(\phi)$ whenever $h$ is, hence $\text{Ker}(\phi)$ is a normal subgroup.
\end{lemma}
\hfill
\begin{theorem}
    All normal subgroups are the kernel of some homomorphism.\\
    \\
    Proof: Let $K$ be a normal subgroup of $G$, then the left cosets $gK$ partition $G$ since if $h\in g_1K$ and $h\in g_2K$, then $h=g_1k_1 = g_2k_2$ for some $k_1,k_2\in K$, and then we have $g_2^{-1}g_1=k_2k_1^{-1}\in K$, which lets us conclude that $g_2K = g_2(k_2k_1^{-1})K=g_2(g_2^{-1}g_1)K=g_1K$. \\
    \\
    We then get a well defined operation $gK\cdot hK = ghK$, since if $gK=g'K$, $g'=gk_g$ for some $k_g\in K$. Then $g'K\cdot hK = gk_ghK$, and since $K$ is normal, $h^{-1}kh=k'\in K$ for all $k\in K$ and for some $k'\in K$, hence $gk_ghK=ghk_g'K=ghK$ and the operation is well defined.\\
    \\
    Call this new group of left cosets $G/K$, and take the quotient map $\pi:G\to G/K$ which takes $g\to gK$. This map is a homomorphism and $Ker(\pi)=K$
\end{theorem}
\hfill
\begin{corollary}
    If $\phi : G\to G'$ is onto, then $G/\text{Ker}(\phi)\cong G'$
\end{corollary}
\hfill
\section{\textbf{LaGrange's Theorem}}
\hfill
\begin{definition}
    If $H < G$ is a subgroup, then $[G:H]=$ the number of left cosets = index of $H$ in $G$
\end{definition}

\begin{theorem}[LaGrange]
    $|G|=[G:H]|H|$
\end{theorem}
\begin{corollary}
    The order of a subgroup of $G$ divides the order of $G$
\end{corollary}
\hfill
\section{\textbf{Group Actions}}

The action of a group $G$ on a set $A$ is a homomorphism:
$$
\sigma:G\to\text{Aut}(A)
$$
\begin{definition}
    Alternatively, we define a (left) action as a map $\rho:G\times A\to A$ such that:\\
    \\
    1) $\rho(e,a)=a$\\
    2) $\rho(gh,a)=\rho(g,\rho(h,a))$
\end{definition}


\textbf{Why are group actions so great?}\\
\\
Every group acts $\textit{faithfully}$ on some set, i.e. $ga=a$ $\forall a\in A$ implies $g=e$. Therefore every group is a subgroup of a permuation group.




\end{document}
